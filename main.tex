\documentclass[stu,12pt]{apa7}
  \usepackage{times}               % Times New Roman Font Face
  \usepackage[american]{babel}     % Localization
  \usepackage[utf8]{inputenc}      % Input Encoding
  \usepackage{hyperref}            % Hyperlinks
  \usepackage{enumitem}            % Additional Enumeration Environment Settings
  \usepackage{geometry}            % Page Layout
  \usepackage{soul}                % Text Highlighting
  \usepackage{graphicx}            % Images
  \usepackage{csquotes}            % Quoting Environment
  \usepackage{bookmark}            % Required by `csquotes'
  \usepackage{mdframed}            % Colorful Tex-Box Environment
  \usepackage[toc]{appendix}       % Appendix
  \usepackage{fancyhdr}            % Headings and Footers
  \usepackage{xcolor}              % Text Colors
  \usepackage[style=apa, sortcites=true, sorting=nyt]{biblatex}  % Bibliography


  % Bibliography Setup
  % Language Mappings
  \DeclareLanguageMapping{english}{english-apa}
  \DeclareLanguageMapping{american}{american-apa}
  % Bibliography File Path
  \addbibresource{main.bib}


  % Hyperlink Setup
  \hypersetup{
    colorlinks = true,
    urlcolor = blue,
    linkcolor = blue,
    citecolor = blue
  }


  % Page and Text Layout
  \geometry{%
    a4paper,%
    top=1in,%
    bottom=1in,%
    left=1in,%
    right=1in%
  }
  \setlength{\headheight}{15pt}


  % Title Page
  \title{%
    M4D2: Improving YOUR Interpersonal Communication Skills
  }
  \shorttitle{Module 4 Discussion 2}
  \author{Ashton Hellwig}
  \authorsaffiliations{Department of Mathematics, Front Range Community College}
  \course{COM125: Interpersonal Communication}
  \professor{Richard Thomas}
  \duedate{November 28, 2020 23:59:59 MDT}
  \date{\today}
  \lhead{COM125CG1-M4D2}
  \abstract{%
    \textbf{Overview}\\%
    In Module 1 you committed to becoming a better interpersonal communicator by
      focusing on improving two interpersonal communication skills. In each
      subsequent module, we are checking in to see how those efforts are
      proceeding based on your practice sessions. In this module, you will
      report on how your practice has progressed so far.\\%

    You should spend approximately 4 hours on this assignment.%
  }

\begin{document}
  % Title Page
  \maketitle


  \section*{Instructions}
    Take a few moments to reflect upon your nonverbal communication skills.
      In your Main post, reply to the following questions:

    \begin{enumerate}
      \item Do your non-verbals give too much away when communicating?
      \begin{itemize}
        \item Is this an area you need to work on? Why or why not?
      \end{itemize}
      \item As we saw in the movie clip appearing within the topic titled,
        ``The Theme'', Pete and Debbie may be saying one thing, but their body
        language is telling a different story. Tell us about a time when you had
        a similar experience.
    \end{enumerate}


  % Initial Post
  \newpage
  \section{Initial Post}
    This past week was actually difficult for me to attempt at practicing
      many interpersonal communication skills, as I was on ``vacation'' in
      Los Angeles, California visiting my Mom, Dad, and Brother for the
      Thanksgiving Holiday. While in LA, a \textit{LOCKDOWN} order was in effect
      causing me to be unable to socialize or experience much of LA outside of
      my parent's house and food delivery and limited pick-up.

    In a way, the fact that everyone is wearing masks in public is sort of
      making nonverbal interpersonal communication skills more relevant than
      ever. I have never seen (or maybe I just did not notice prior to the
      pandemic) the level of expression people are able to convey with just
      the way they position their eyes and express other body language,
      especially without the ability to rely on noticing how somone positions
      their lips (smiling, frowning, afraid, shock, et cetera). Through just
      someones eyes it is easier for us to understand if and be empathetic
      towards their emotion be it happiness, sadness, or disgust, and
      exhaustion.

    Many intelligence agencies use something called the \textit{Facial Action
      Coding System} in order to have agents act as rudimentary ``human
      lie-detectors''. The system, at its basic idea, assines certain values
      to different movements of facial muscles to be assigned when observing
      someone else. There are over around 46 basic ``action units'' (movements
      of the facial muscles) which are then assigned a letter value to
      illustrate their intensity --- clearly, a lot can be told from just
      ``simply'' reading someone's face
      \parencite[pp. 207--209]{cohn_observer-based_2007}.

  % Replies
  %! TEX root=../main.tex

\section{Responses}
  \subsection{Response 1}
    \begin{quotation}
      \texttt{[\ldots]}
      I, personally, don’t think my non-verbals give too much away when I am
        communicating. I am pretty good at keeping myself in check with my body
        language, and displaying that I am engaged in the conversation. I will
        say though, that a lot of the time when I am conversing with others, I
        tend to cross my arms. This is not because I am annoyed or
        disinterested, it is just a comfortable position for me, and one I
        resort to. However, this could be received negatively by the person I
        am having a conversation with, and make me come across as rude. I could
        definitely work on this more, so my non-verbals don’t say one thing,
        when I am actually feeling the opposite. One example of this was when I
        was speaking with my friend, I crossed my arms, but soon after I noticed
        a shift in his attitude. I realized that he probably got the impression
        that I was annoyed, angry, not interested in the conversation, or
        somewhere in between. I immediately changed my body language to show
        that I was in fact engrossed in the conversation.
    \end{quotation}

    \paragraph{This is a response to Kristyna Sekera on Post ID 43560713}
      Placeholder.

  %! TEX root=../main.tex

\subsection{Response 2}
  \begin{quotation}
    Placeholder.
  \end{quotation}

  \paragraph{This is a response to FIRST LAST on Post ID 00000000}
    Placeholder.



  % Bibliography
  \printbibliography[]
\end{document}
