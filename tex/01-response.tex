%! TEX root=../main.tex

\section{Responses}
  \subsection{Response 1}
    \begin{quotation}
      \texttt{[\ldots]}

      I, personally, don’t think my non-verbals give too much away when I am
        communicating. I am pretty good at keeping myself in check with my body
        language, and displaying that I am engaged in the conversation. I will
        say though, that a lot of the time when I am conversing with others, I
        tend to cross my arms. This is not because I am annoyed or
        disinterested, it is just a comfortable position for me, and one I
        resort to. However, this could be received negatively by the person I
        am having a conversation with, and make me come across as rude. I could
        definitely work on this more, so my non-verbals don’t say one thing,
        when I am actually feeling the opposite. One example of this was when I
        was speaking with my friend, I crossed my arms, but soon after I noticed
        a shift in his attitude. I realized that he probably got the impression
        that I was annoyed, angry, not interested in the conversation, or
        somewhere in between. I immediately changed my body language to show
        that I was in fact engrossed in the conversation.
    \end{quotation}

    \paragraph{This is a response to Kristyna Sekera on Post ID 43560713}
      Lovely and excellently thorough post, Kristyna! My attention was caught
        by where you began to talk about how you try to keep your body language
        in check in order not to give yourself away too much during informal
        and casual conversation. I feel that those individuals with perhaps a
        lot of experience in dealing with people and reading body language
        would be able to detect when someone else is attempting to consciously
        adjust their own body language. Would you say that you have notice this
        in anyone else? When I have spoken to someone about a situation in which
        they may have done something wrong or are trying to hide something, what
        that person may feel is a ``calm collected exterior'' could come off
        to the other person as anxious or ansy. Though there are many people
        I have interacted with that are just as \textbf{stoic} as the day
        is long!
