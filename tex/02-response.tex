%! TEX root=../main.tex

\subsection{Response 2}
  \begin{quotation}
    I am working on nonverbal communication and confidence. Since last module,
      I actually worked on my two skills three times yesterday (11/23/2020). I
      worked on these skills at work with some success and failure. I was able
      to exhibit nonverbal communication when communicating at work; I just
      focused on my body language and speed of talking as well as tone of voice.
      However, confidence was something I struggled with even though I told
      myself positive statements and focused on body language to improve my
      confidence. I don’t have a huge success story, but something happened
      that did not go well while working on these skills. I had to deliver some
      negative news to a parent, and ended feeling super insecure and not
      confident at all. I think this contributed to their negative response.

    I think people misinterpret my non-verbals quite frequently when I am
      communicating. So I can’t really say if they always do or don’t give away
      too much, it kind of just depends on the situation and the person. This
      is an area I need to work on since my non-verbal signals do not always
      match how I am feeling, or they exaggerate how I am really feeling. A time
      body language was telling a different story, was when I was working with a
      client. I asked the client if they were feeling okay, and they said yes.
      However, their tone of voice sounded sad, their shoulders were slumped,
      their head was down, and they didn’t want to really talk to me.
  \end{quotation}

  \paragraph{This is a response to Ashley Nefflen on Post ID 43557683}
    I, too, am working on confidence! I wish you the best of luck because it
      is not all that easy to put yourself out there without knowing the outcome
      of how others will view you afterwards!

    I wanted to ask about the situational differences you noticed in other's
      understanding of your nonverbal communication. You said it depended on
      the situation, but I was wondering: does it depend more on the situation
      or the status of the relationship between the other person and oneself?
      I had known people in the past where no matter the situation, even after
      repeat occurrences, could simply \textit{not} get nonverbal communication
      cues I gave off! Is this similar in your experience, or are the same
      people more understanding of your nonverbal communication cues in
      differing scenarios?
